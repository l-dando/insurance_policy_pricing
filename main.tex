\documentclass{article}
\usepackage[utf8]{inputenc}
\usepackage{amsmath,amssymb}
\usepackage[top=2cm, bottom=2cm, left=2cm, right=2cm]{geometry}
\usepackage[font=bf,figurename=Fig.,justification=centering]{caption}
\usepackage{graphicx,wrapfig,subcaption,adjustbox,tikz,float,biblatex}
\usepackage{url}

\bibliography{references.bib}


\pagenumbering{roman}

\title{GLMs within Insurance Policy Pricing}
\author{Luke Dando}
\date{\today}

\begin{document}

\maketitle
\tableofcontents
\pagebreak
\pagenumbering{arabic}

\section{What is risk?}
Risk is the chance that something harmful or unexpected could occur regarding a held policy. This might involve loss, theft, or damage of valuable property and belongings, or it may involve someone being injured. From a statistical point-of-view risk can be defined as:
\begin{equation}
    \tau = \mathbb{E}\left[\frac{L}{e}\right].
\end{equation}
Here, $L$ is the loss and $e$ is the exposure of the policy. If we now assume that the frequency of a claim is independent from the severity of it, we find:
\begin{align}
    \tau &= \mathbb{E}\left[\frac{L}{e}\right] \\
    &= \mathbb{E}\left[\left.\frac{L}{N}\right\vert\right]
\end{align}

\end{document}
